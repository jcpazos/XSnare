\section{Implementation}

We have implemented our browser extension in Firefox 67.0.2. Our signatures are currently stored in a local JavaScript file in the extension package. Injection point sanitization is done with DOMPurify \cite{10.1007/978-3-319-66399-9_7}. This library is described by its creators as a "DOM-only, super-fast, uber-tolerant XSS sanitizer for HTML, MathML and SVG". The Mozilla community cites it as an useful tool for "safely inserting external content into a page" \cite{safecontent}. While it offers a lot of configurability and hooks, we use the default functionality, with satisfying results, as described in Section 4.

As previously mentioned, our detector loads signatures and finds injection points in the document. However, there might be a large number of signatures which don't need to be loaded for a specific website. For example, if several signatures are designed for pages running a WordPress plugin, then the extension need not check any of these for a site which is not running WordPress at all. On the other hand, if a site is running WordPress, we might have to check all signatures meant for WordPress, but not others. Therefore, when loading signatures, we proceed in a manner similar to a decision tree. The detector first probes the page to identify the underlying framework (we call this the 'software' in our signature language). These are usually found by hints in the document HTML. Probes are framework-specific, and as such, need to be encoded in some way so that the detector can run them. There are two approaches for this: the detector completely takes care of this, and needs to be maintained for changes in frameworks and future technologies, or, the signature developer additionally specifies a more specialized version detector as part of a probe file. We chose to go with the first option, as this provides a greater ease of use for signature developers. In our prototype implementation, hard-coding probes detector did not imply a substantial amount of work, however, as more signatures are written and more applications are required to be included, this can become an arduous task. We believe the second option can be desirable and would not be a terrible burden for signature developers: for example, the widely popular network mapping tool Nmap \cite{nMap} uses probes in a similar manner, and these are kept in a modifiable file so that advanced users can have more expressibility. After running these probes, the detector loads signatures for the specific software (e.g. all signatures for WordPress web sites). At this point, we filter out the ones that do not apply to the current page.

Finally, we apply version identification. Our objective for versioning is that our signatures don't trigger any false positives on websites running patched software. We found this to be one of the harder aspects of signature loading. In WordPress, for example, many of the plugins do not update their file names with the latest versions, or do not include them at all, and thus, this information is often hard to come by from the client-side perspective. In the case of WordPress, the wp-admin/plugins.php subpath contains information about all currently active plugins on the site. Unfortunately, this information is only available to admins of the site. While this might not be the bulk of users, it is, on the other hand, the bulk of disclosed CVEs, as described in Section 4. Furthermore, we believe that even if we load a signature when the application has already been fixed at the server-side, it will often preserve the page's functionality, as many of the CVEs describe XSS which happens as a result of unsanitized input that was not meant to be JavaScript code regardless. Motivated by this observed behaviour, our mechanism follows a series of more accurate but less applicable version identifiers: first, we apply general-purpose version probes, like the one described for WordPress (these are maintained in a similar way to software probes, hard-coded in the detector logic). If these are not successful, the signature language provides functionality for version identification in the HTML through regex. If the developer considers version information to be unavailable through the HTML, the version in the signature is left blank and the detector applies the signature patch regardless of version, as we can not be sure the page is running patched software.

Some of the exploits manifest themselves through dynamically loaded files. For example, CVE-2018-7747 had XSS triggered when the user loaded information stored in the plugin's database after clicking on an element of the page. Since this was not loaded with the original HTML, it came as a response to an Ajax request. Our signature language provides functionality to protect against these kinds of exploits, as shown in the example below:

 \lstset{basicstyle=\small}
\begin{lstlisting}
url: 'wp-admin/admin.php?page=caldera-forms',
...
listenerData: {
listenerType: 'xhr',
listenerMethod: 'POST',
listenerUrl: 'wp-admin/admin-ajax.php'
},
typeDet: 'multiple-unique',
description: '',
sigOccurrence: 'unique',
endPoints: [
	['<tr id="entry_row_3">', 
	'<button class="hidden button button-small ...'],
	['<tr id="entry_row_2">', 
	'<button class="hidden button button-small ...']
	...
]
\end{lstlisting}

This signature describes an exploit on a WordPress site running the Caldera Forms plugin. The XSS occurs in the specified url. The listenerData attribute defines an extra listener to attach in the background page of the extension. In this case, the page listens for an XHR, specifically done as a POST to the specified subdomain listenerUrl. The rest of the information is similar to a regular signature, as it will execute the filter and sanitize the response if necessary, according to the specified endpoints. The background page knows to only filter such requests originated from the correct web page. The type of resource to listen for is taken as specified by the webRequest API. 

While our prototype currently does not have a centralized database for signatures, we envision that a trusted entity can maintain this database, such as those that currently maintain CVE databases, and will have to be audited so that it is not filled with false signatures and existing ones are not compromised.
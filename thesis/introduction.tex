\section{Introduction}

Cross-site scripting (XSS) is still one of the most dominant web vulnerabilites. In 2017, a report showed that 50\% of websites contained at least one XSS vulnerability \cite{Acunetix}. Even though many countermeasures have been researched to combat these issues, many of them lack widespread deployment, and so have been unable to protect users. Many of these defenses leverage server-side techniques, along with browser modifications \cite{Jim:2007:DSI:1242572.1242654,Nadji:2009}; or require additional developer effort \cite{10.1007/978-3-319-66399-9_7}. Still, others disable client-side functionality \cite{Noscript,Snyder:2017:MWD:3133956.3133966}, sometimes rendering websites unusable. We believe many of these solutions have not seen widespread adoption because they simply are not practical: developers might not be willing, or might not have the resources or expertise available to implement them. Furthermore, even when enough information is available for a developer, and they are able to fix these vulnerabilities, many website administrators won't deploy fixes immediately: a 2016 study found that 61\% of WordPress websites were running a version with known security vulnerabilities \cite{Sucuri}, and another found that 30.95\% of Alexa's top 1 Million sites run a vulnerable version of WordPress \cite{wpwhitesecurity} TODO: how many of these can run without JS?. As the number of websites using client-side technologies continues to increase (a study showed that as of 2012, almost 100\% of the Alex top 500 sites were using JavaScript \cite{Stock:2017:WTI:3241189.3241265}), users are effectively left at the mercy of developers, without tools that both allow them to protect themselves and browse the web worry-free.

Our work focuses on WordPress as a study platform, we look at recent CVEs related to WordPress plugins. While this may seem restrictive, there are several reasons why this is a worthwhile endeavour:
\begin{itemize}
	\item WordPress powers 25\% of all websites according to a recent survey. Furthermore, 30.3\% of the Alexa top 1000 sites use WordPress \cite{w3techs}. Thus, we can be confident that our study results will hold true to for the average user.
	\item Due to its user popularity, it is also heavily analysed by security experts, as has been previously stated. There are currently 286 CVEs related to WordPress in the CVE Details database \cite{cvedetails}. Plugins, specifically, are an important part of this issue, 52\% of the vulnerabilities reported by WPScan are caused by WordPress plugins \cite{wpscan}.
	\item Due to the open source nature of WordPress plugins, we can easily analyze both the client-side HTML, as well as the server-side code that generated it, and use this to reach conclusions about the design of our solution.
	
\end{itemize}

 Even though our study has not focused on other sites, our approach is not limited to a specific framework, and we believe it should generalize to arbitrary webpages, as long as we have a pre-existing notion of a webpage's contents.

To provide users with the means to protect themselves in the absence of control over servers, we strongly believe a client-side solution must be delivered. A number of existing solutions also suffer from a high rate of false-positives and false-negatives, due to the lack of information available at the layers they operate at (e.g. blacklisting in the NoScript browser extension). In contrast, we posit that the DOM is the right place to interpose for the purpose of mitigating against these attacks, since we have the full picture at that point. TODO: expand on this, explain why DOM is the right place.

 Our system consists of three main components: a trusted Firefox extension for interposing between the application and the DOM, a periodically updated local database which maintains exploit definitions and descriptions of the steps needed to be followed by the extension, and finally, a declarative language for defining exploits, expressive enough for an user to be precise about which parts of the HTML are vulnerable. 



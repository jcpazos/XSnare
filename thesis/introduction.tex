\section{Introduction}

Cross-site scripting (XSS) has long been one of the most dominant web vulnerabilites. In 2017, a report showed that 50\% of websites were vulnerable to XSS attacks \cite{Acunetix}. Even though many countermeasures have been developed to combat these issues, many of them lack widespread deployment, and so have been unable to protect users. Many of these defenses leverage server-side techniques, along with browser modifications \cite{Jim:2007:DSI:1242572.1242654,Nadji:2009}; or require additional developer effort \cite{10.1007/978-3-319-66399-9_7}. Still others disable client-side functionality \cite{Noscript,Snyder:2017:MWD:3133956.3133966}, sometimes rendering websites unusable. We believe many of these solutions have not seen widespread adoption because they simply are not practical: developers might not be willing, or might not have the resources or expertise available to implement them. Furthermore, even when enough information is available for a developer, and they are able to fix these vulnerabilities, many website administrators won't benefit from these immediately: according to WordPress, only 65.9\% of websites running WordPress have currently updated to the latest version \cite{WPStats}. As the number of websites using client-side technologies continues to increase (a study showed that as of 2012, almost 100\% of the Alex top 500 sites were using JavaScript \cite{Stock:2017:WTI:3241189.3241265}), users are left more exposed than ever to client-side vulnerabilities.

To provide users with the means to protect themselves, a client-side solution must be delivered. The aforementioned solutions also suffer from an increased rate of false-positives and false-negatives, due to the lack of information available at these layers. In contrast, the DOM is the right place to interpose for the purpose of mitigating against these attacks, since we have the full picture at that point. Our system consists of three main components: a trusted Firefox extension for interposing between the application and the DOM, an automatically updating local database which maintains exploit definitions and descriptions of the steps needed to be followed by the extension, and finally, a declarative language for defining exploits, expressive enough for an user to be precise about which parts of the HTML are vulnerable. 



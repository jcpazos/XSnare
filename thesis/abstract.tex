\begin{abstract}

We present \sys, a fully client-side \ac{XSS} solution, implemented as
a Firefox extension. Our approach takes advantage of available
previous knowledge of a web application's HTML template content, as
well as the rich context available in the DOM to block XSS
attacks. \sys prevents \ac{XSS} exploits by using a database of
exploit descriptions, which are written with the help of previously
recorded CVEs. CVEs for XSS are widely available and are one of the
main ways to tackle zero-day exploits. \sys effectively singles out
potential injection points for exploits in the HTML and sanitizes
content to prevent malicious payloads from appearing in the DOM;
displaying a secured version of the site, even if is exploited.

\sys can protect application users
before application developers release patches and before server
operators apply them.

We evaluated \sys on 81 recent CVEs related to XSS
attacks, and found that it defends against 93.8\% of these
exploits. To the best of our knowledge, \sys is the first protection
mechanism for XSS that is application-specific, and based on publicly
available CVE information.  We show that \sys{}'s specificity protects
users against exploits which evade other, more generic, anti-XSS
approaches.

Our performance evaluation shows that our extension's overhead on web
page loading time is less than 10\% for 72.6\% of the sites in the Moz
Top 500 \cite{top500} list.

\end{abstract}

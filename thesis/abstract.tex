\begin{abstract}

We present XSnare, a fully client-side cross-site scripting (XSS)
solution, implemented as a Firefox extension. Our approach takes
advantage of available previous knowledge of a web application's
HTML template content, as well as the rich context available in the
DOM to block XSS attacks. It uses a database of exploit
descriptions, which are written with the help of previously recorded
CVEs, to prevent them. CVEs for XSS are widely available and are
one of the main ways to tackle zero-day exploits. We effectively
single out potential injection points for exploits in the HTML and sanitize content to prevent malicious payloads from appearing in the DOM.

XSnare is a framework for emulating server-side software sanitization on the application client-side. It can protect application users before application developers release patches and before server operators apply them.

We evaluate the applicability of our approach by studying the latest
100 CVEs related to XSS attacks in WordPress, and find that our tool
defends against 93.4\% of these exploits. To the best of our knowledge,
our work is the first to provide a protection mechanism for XSS that is
application specific, and based on publicly available CVE information.
We show that XSnare's specificity protects users against exploits which evade other, more generic, anti-XSS approaches.

Our performance evaluation shows that our extension's overhead on web page loading time is less than 10\% for 72.6\% of the sites studied.

\end{abstract}

\section{Related Work}
We classify existing work into multiple categories: client-side, server-side, browser built-in, and hybrid: a combination of these.

\noindent \textbf{Server-side techniques.} 
In addition to existing parameter sanitization techniques,
taint-tracking has been proposed as a means to consolidate
sanitization of vulnerable parameters
\cite{Xu:2006:TPE:1267336.1267345,DBLP:conf/sec/Nguyen-TuongGGSE05,Pietraszek:2005:DAI:2146257.2146267,Bisht:2008:XPD:1428322.1428325}. These
techniques are complementary to ours, and provide an additional line
of defence against \ac{XSS}. However, many of them rely on the
client-side rendering to maintain the server-side properties, which
will not always be the case.

\noindent \textbf{Client-side techniques.} 
DOMPurify~\cite{10.1007/978-3-319-66399-9_7} presents a robust
\ac{XSS} client-side filter. The authors
argue that the DOM is the ideal place for sanitization to occur. While
we agree with this view, their work relies on application developers
to adopt their filter and modify their code to use it. Thus, we have partly automated this step by including it as our default sanitization function.

Jim et al. \cite{Jim:2007:DSI:1242572.1242654} present a method to
defend against injection attacks through Browser-Enforced Embedded
Policies. This approach is similar to ours, as the policies specify
prohibited script execution points. However, this again relies on application developers knowing where their code might be vulnerable. Furthermore, browser modifications are required to benefit from it, which can cause issues of portability and backwards compatibility. Similarly, Hallaraker and Vigna \cite{Hallaraker:2005:DMJ:1078029.1078861} use a
policy language to detect malicious code on the client-side. Like \sys, they make use of signatures to protect against known types of
exploits. However, unlike our approach, their signatures are not
application-level, and there is no model for signature
maintenance. Furthermore, there is no evaluation on the efficacy of
their signatures.

Snyder et al.~\cite{Snyder:2017:MWD:3133956.3133966} report a study in which
they disable several JavaScript APIs and test the number of websites
that are do not work without the full functionality of the APIs. This approach increases security due to vulnerabilities present in several
JavaScript APIs, however, we believe disabling API functionality
should only be used as a last resort. We initially thought of blocking exploits in a similar manner, but we would lose precision during sanitization, as described in \autoref{firefox_extension}.

\noindent \textbf{Browser built-in defences.}  Browsers are equipped
with several built-in defences. We previously described XSS
Auditor ~\ref{introduction}), the other two important ones are Content
Security Policy and the Same-origin policy.

Content Security Policy (CSP)~\cite{CSP} has been widely adopted and
in many cases provides developers with a reliable way to protect
against \ac{XSS} and Cross-site request forgery (CSRF)
attacks. However, CSP requires the developer to know which scripts
might be malicious.

Same-origin policy~\cite{SOP} is another useful security mechanism for
protection against \ac{XSS} and CSRF. This policy restricts how a
document or script loaded from one origin can interact with a resource
from another origin. However, if the attack is injected in the same
website the attacker intends to compromise, this will not defend
against it. As with other approaches these browser policies are complementary to XSnare.

\noindent \textbf{Client and server hybrids.}
XSS-Dec~\cite{Sundareswaran:2012:XHS:2352970.2352994} uses a proxy which keeps track of an encrypted version of the server's source files, and applies this information to derive exploits in a page visited by the user. This approach is similar to ours, since we assume previous
knowledge of the clean HTML document. Furthermore, they user anomaly-based and signature-based detection to prevent attacks. However, there is no mention of signature maintenance. In a way, our system offloads all this functionality to the client-side, without the need of any server-side information.


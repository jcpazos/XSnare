\section{Limitations and Future Work}
\textbf{Generalizability.} Our study has only covered WordPress websites. While many websites, in particular ones that use any kind of CMS might share similar structures to the ones we studied, it is clear that the open source nature and availability of WordPress code and its plugins might have made our assumption of full knowledge of the HTML too strong. We acknowledge that this assumption will not always hold true, however, many websites will still be able to benefit from our approach.

\textbf{Scope of study.} Our current CVE study has only covered 100 CVEs, 23 of which had to be discarded in our result analysis. We intend to cover more in the future to have a better representation of WordPress websites and plugins and the web as a whole. 

\textbf{Current implementation.} There are several aspects of injections that we have not been able to test as a result of the CVEs not exploiting these, like attacks through images. Our prototype implementation lacks several performance enhancements which would greatly benefit the overhead caused by the extension, like parallelizing signature loading and sanitization, as well as using more efficient data structures to store signatures. 

\textbf{False positives and false negatives.} Due to the nature of our approach, it is nigh impossible to completely get rid of false positives. While we concluded that the rate of false positives is ver low for loaded signatures, these can still occur when a signature is correctly loaded, since the sanitization might get rid of elements that were intended to be there. We have previously discussed the ability to reproduce the developer's intention with regards to when scripts should be able to run, however, this won't always be possible, as there will be cases where there are injection points where non-malicious JavaScript is allowed, and thus, our system will have false positives. Furthermore, since we rely on handwritten signatures to defend against attacks, it is possible that not every single injection point of every website will be covered, and so we will also have false negatives. In the future, we intend to study the rate of false positives and negatives in our approach and compare it to previous work. Faulty sanitization could be reduced by implementing our sanitization methods as a lexer instead of the declarative version we currently have: The signature developer would ideally provide the allowed behaviour in a given injection point, and the detector would check the injection content against the signature-specified behaviour, providing a proof of whether the content could have been generated by following the rules, and if not, eliminating any content that could not be have been generated from them.

\textbf{Usability.} A main aspect of our work is its increased potential for usability and adoption from both an user's perspective that installs the extension to defend themselves against XSS, and a signature developer that has to write the database descriptions according to a known CVE. Future work could focus on usability studies related to both of these components.

\textbf{Protection against CSRF.} While our work has only focused on XSS, we believe we can easily adapt our network filter to defend against CSRF exploits as well. Using a similar signature language as the one for XSS, a signature developer could specify pages with potential vulnerabilities and enforce network requests that can not exploit such vulnerabilities. In some scenarios, it might be useful to track DOM events, e.g. user clicks to protect against clickjacking attacks. This kind of information is not available to the network; however, the extension could install a script on the page to monitor these actions. 

\textbf{Dealing with an increasing amount of signatures.} Our naive approach at signature filtering (once a probe for a framework has passed) has been effective with our current amount of signatures. However, as the number of signatures increases, and more types of sites are covered, the performance impact may increase drastically. As of September 2019, CVE Details had 14894 XSS vulnerabilities in its database \cite{vulnbytype}. This would mean that our current signatures only cover 0.5\% of all existing XSS vulnerabilities. In reality, the figure would not be as small, since many of these vulnerabilities are old and the software to which they applied has already been updated, making the probability of encountering an exploit very low. Regardless, more efficient approaches at searching and filtering, as well as for the data structures used in the signature database could be applied to maintain a low performance overhead. (TODO: someone might know or have referenced to what firewalls do?)

\section{Limitations and Future Work}
\textbf{Generalizability.} Our study has only covered WordPress websites. While many websites, particular ones that use any kind of CMS might share similar structures to the ones we studied, it is clear that the open source nature of availability of WordPress code and its plugins might have made our assumption of full knowledge of the HTML too strong. We acknowledge that this assumption will not always hold true, however, many websites will still be able to benefit from our approach.

\textbf{Scope of study.} Our current study has only covered 100 CVEs, 21 of which had to be discarded in our result analysis. We intend to cover more in the future to have a better representation of WordPress websites and plugins and the web as a whole. 

\textbf{Current implementation.} The DOM Firewall has only been manually tested by specially crafted signatures. We are still in the process of refining the signature language and releasing a general framework for signature descriptions and the process of uploading and downloading to the firewall database.

\textbf{False positives and false negatives.} Due to the nature of our approach, it is nigh impossible to completely get rid of false positives. We have previously discussed the ability to reproduce the developer's intention with regards to when scripts should be able to run, however, this won't always be possible, as there will be cases where there are injection points where non-malicious JavaScript is allowed, and thus, our system will have false positives. Furthermore, since we rely on handwritten signatures to defend against attacks, it is possible that not every single injection point of every website will be covered, and so we will also have false negatives. In the future, we intend to study the rate of false positives and negatives in our approach and compare it to previous work. 

\textbf{Performance.} We haven't been able to evaluate our extension's performance. The added filtering and auditing of the network responses and the DOM will incur some overhead in the a website load times, but we don't expect this to be too detrimental, as the browser APIs provide fast methods to filter requests and interpose on event loads.

\textbf{Usability.} A main aspect of our work is its increased potential for usability and adoption from both an user's perspective that installs the extension to defend themselves against XSS, and a signature developer that has to write the database descriptions according to a known CVE. Future work could focus on usability studies related to both of these components.

\appendix
\section{Signature Language Specification} \label{appendix:language_specification}
We provide a description of our signature language, in particular in the context of WordPress:
\begin{itemize}
	\item
	url: If the exploit occurs in a specific URL or subdomain, this is defined as a string, e.g.
	\url{/wp-admin/options-general.php?page=relevanssi\%2Frelevanssi.php}, otherwise null.
	\item
	software: The software framework the page is running if any, e.g. WordPress. A hand-crafted page
	might not have any identifiable software.
	\item
	softwareDetails: If running any software, this provides further information about when to load a signature. For WordPress, these are plugin names as depicted in the HTML of a page running such plugin.
	\item
	version: The version number of the software/plugin/page. This is used for versioning as described earlier.
	\item 
	type: A string describing the signature type. A value of "string" describes a basic signature. A value of 'listener' describes a signature which requires an additional listener in the background page for network requests.
	\item 
	sanitizer: A string with one of the following values: "DOMPurify", "escape", and "regex". This item is optional, the default is DOMPurify.
	\item
	config: The config parameters to go along with the chosen sanitizer, if necessary. For "DOMPurify", the accepted values are as defined by the DOMPurify API (i.e DOMPurify.sanitize(dirty, config). For "escape", an additional escaping pattern can be provided. For "regex", this should be the pattern to match with the injection point content.
	\item
	typeDet: A string determining an injection's occurrence, either "single" or "multiple". As described in Section 2.4, this specifies whether there are several injection points in the HTML.
	\item
	endPoints: An array of startpoint and endpoint tuples
	\item 
	endPointsPositions: An array of integer tuples. These are optional but useful when the one of the endPoints HTML are used throughout the whole page and appear a fixed number of times. For example: if an injection ending point happens on an element <h3 class='my-header'>, this element might have 10 appearances throughout the page. However, only the 4th is an injection ending point. The signature would specify the second element of the tuple to be 7, as it would be the 7th such item in a regex match array (using 1-based indexing), counting from the bottom up. For ending points, we have to count from the bottom up because the attacker can inject arbitrarily many of these elements before it, and vice versa for starting points.
\end{itemize}

Additionally, if the value of type is 'listener', the signature will have an additional field called listenerData. Similarly to a regular signature, this consists of the following pieces of information:
\begin{itemize}
	\item 
	listenerType: The type of network listener as defined by the WebRequest API (e.g. 'script, 'XHR', etc.)
	\item
	listenerMethod: The request's HTTP method, for example "GET" or "POST".
	\item
	url: the URL of the request target.
\end{itemize}
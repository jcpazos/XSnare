\section{Conclusion}

Users cannot depend on administrators to patch vulnerable server-side
software or for developers to adopt best practices to mitigate XSS
vulnerabilities. Instead, users should protect themselves with a
client-side solution. In this paper we described the design,
implementation, and evaluation of one such client-side approach in
\sys.
%
\sys prevents \ac{XSS} exploits by using a database of exploit
signatures and by using a novel mechanism to detect XSS exploits in a
browser extension.
%
We evaluated our approach through a study of 80 CVEs in which we
showed that our tool defends against 94.2\% of the exploits.

%% We have presented \sys, a fully client-side protection mechanism
%% against XSS.
%% \sys has many benefits over existing
%% systems, as well as being complementary to many of them.
%% The \sys firewall
%% architecture allows users to protect themselves in the face
%% of an ever-increasing number of potential attacks and attack vectors,
%% with little additional effort required for a knowledgeable security
%% analyst when taking into consideration the existing vulnerability
%% detection work flow.



%%  we conducted showed that the provided API for
%% sanitization and injection point detection is effective, as 93.75\% of
%% these vulnerabilities could be defended with \sys. Our \sys prototype
%% meets the required performance goals to not be detrimental to a user's
%% web experience, as web page load times increase by less than 10\% on
%% 72.6\% of sites.
